\documentclass{report}
\usepackage{amsfonts, amsmath, amssymb, hyperref}
\renewcommand{\chaptername}{}
\begin{document}

\tableofcontents

\chapter{Brusselator}

The Brusselator is characterized by the reactions

\begin{align}
  &A \rightarrow X \\
  &2X + Y \rightarrow 3X \\
  &B + X \rightarrow Y + D \\
  &X \rightarrow E
\end{align}

and the rate equations are

\begin{align}
  \frac{\mathrm{d}}{\mathrm{d}t} \{X\} &= \{A\} + \{X\}^2\{Y\} - \{B\}\{X\} - \{X\} \\
  \frac{\mathrm{d}}{\mathrm{d}t} \{Y\} &= \{B\}\{X\} - \{X\}^2\{Y\}
\end{align}

%%%%%%%%%%%%%%%%%%%%%%%%%%%%%%%%%%%%%%%%%%%%%%%%%%%%%%%%%%%%%%%%%%%%%%%%%%%%%%%

\chapter{Damped spring}

The damped spring is described by the simple equation

\begin{equation}
  m\frac{\mathrm{d}^2x}{\mathrm{d}t^2} = -kx + -c\dot{x}
\end{equation}

where $k$ is the spring constant and $c$ is the damping coefficient. We can trivially write this as the following 1st-order ODEs:

\begin{align}
  \frac{\mathrm{d}x}{\mathrm{d}t} &= v \\
  \frac{\mathrm{d}v}{\mathrm{d}t} &= -kx + -cv
\end{align}

%%%%%%%%%%%%%%%%%%%%%%%%%%%%%%%%%%%%%%%%%%%%%%%%%%%%%%%%%%%%%%%%%%%%%%%%%%%%%%%

\chapter{Double pendulum}

\section{Lagrangian}

\begin{equation}
  L = \frac{1}{6} ml^2 \left[ \dot{\theta}^2_2 + 4\dot{\theta}^2_1 + 3 \dot{\theta}_1 \dot{\theta}_2 \cos(\theta_1 - \theta_2) \right] + \frac{1}{2} mgl(3 \cos\theta_1 + \cos\theta_2)
\end{equation}

\section{Equations of motion}

The momenta are
\begin{align}
  p_{\theta_1} &= \frac{\partial L}{\partial \dot{\theta}_1} = \frac{1}{6} ml^2 \left[8 \dot{\theta}_1 + 3 \dot{\theta}_2 \cos(\theta_1 - \theta_2) \right] \\
  p_{\theta_2} &= \frac{\partial L}{\partial \dot{\theta}_2} = \frac{1}{6} ml^2 \left[2 \dot{\theta}_2 + 3 \dot{\theta}_1 \cos(\theta_1 - \theta_2) \right]
\end{align}
which can be inverted to
\begin{align}
  \dot{\theta}_1 &= \frac{6}{ml^2} \frac{2 p_{\theta_1} - 3 \cos(\theta_1 - \theta_2) p_{\theta_2}}{16 - 9 \cos^2(\theta_1 - \theta_2)} \\
  \dot{\theta}_2 &= \frac{6}{ml^2} \frac{8 p_{\theta_2} - 3 \cos(\theta_1 - \theta_2) p_{\theta_1}}{16 - 9 \cos^2(\theta_1 - \theta_2)}
\end{align}
Finally,
\begin{align}
  \dot{p}_{\theta_1} &= \frac{\partial L}{\partial\theta_1} = - \frac{1}{2} ml^2 \left[\dot{\theta}_1 \dot{\theta}_2 \sin(\theta_1 - \theta_2) + 3 \frac{g}{l} \sin\theta_1 \right] \\
  \dot{p}_{\theta_2} &= \frac{\partial L}{\partial\theta_2} = - \frac{1}{2} ml^2 \left[- \dot{\theta}_1 \dot{\theta}_2 \sin(\theta_1 - \theta_2) + \frac{g}{l} \sin\theta_2 \right]
\end{align}

%%%%%%%%%%%%%%%%%%%%%%%%%%%%%%%%%%%%%%%%%%%%%%%%%%%%%%%%%%%%%%%%%%%%%%%%%%%%%%%

\chapter{Duffing equation}

The Duffing equation is given by the second-order ODE
\begin{equation}
  \frac{\mathrm{d}^2x}{\mathrm{d}t^2} + \delta \frac{\mathrm{d}x}{\mathrm{d}t} + \alpha x + \beta x^3 = \gamma \cos(\omega t)
\end{equation}
which we can cast into the first-order ODEs:
\begin{align}
  \frac{\mathrm{d}x}{\mathrm{d}t} &= v \\
  \frac{\mathrm{d}v}{\mathrm{d}t} &= \gamma \cos(\omega t) - \delta v - \alpha x - \beta x^3
\end{align}

%%%%%%%%%%%%%%%%%%%%%%%%%%%%%%%%%%%%%%%%%%%%%%%%%%%%%%%%%%%%%%%%%%%%%%%%%%%%%%%

\chapter{Lorenz system}

\begin{align}
  \frac{\mathrm{d}x}{\mathrm{d}t} &= \sigma(y - x) \\
  \frac{\mathrm{d}y}{\mathrm{d}t} &= x(\rho - z) - y \\
  \frac{\mathrm{d}z}{\mathrm{d}t} &= xy - \beta x
\end{align}

\begin{itemize}
  \item $\sigma = 10, \beta = 8/3, \rho = 28$
\end{itemize}

%%%%%%%%%%%%%%%%%%%%%%%%%%%%%%%%%%%%%%%%%%%%%%%%%%%%%%%%%%%%%%%%%%%%%%%%%%%%%%%

\chapter{Lotka-Volterra equations}

%%%%%%%%%%%%%%%%%%%%%%%%%%%%%%%%%%%%%%%%%%%%%%%%%%%%%%%%%%%%%%%%%%%%%%%%%%%%%%%

\chapter{Van der Pol oscillator}

%%%%%%%%%%%%%%%%%%%%%%%%%%%%%%%%%%%%%%%%%%%%%%%%%%%%%%%%%%%%%%%%%%%%%%%%%%%%%%%

\chapter{2-D Van der Pol oscillator}

%%%%%%%%%%%%%%%%%%%%%%%%%%%%%%%%%%%%%%%%%%%%%%%%%%%%%%%%%%%%%%%%%%%%%%%%%%%%%%%

\chapter{Symmetric top}

\section{Moments of inertia}

\begin{align}
  I_1 = I_2 &= \frac{3}{20} m \left( r^2 + \frac{l^2}{4} \right) \\
  I_3 &= \frac{3}{m r^2}
\end{align}

\section{Lagrangian}

\begin{equation}
  L = \frac{1}{2} I_1 \left( \dot{\theta}^2 + \dot{\varphi}^2 \sin^2\theta \right) + \frac{1}{2} I_3 \left( \dot{\psi} + \dot{\varphi}\ \cos\theta \right)^2 - mgl \cos\theta
\end{equation}

\section{Equations of motion}

The Euler-Lagrange equation gives
\begin{equation}
  \frac{d}{dt} \frac{\partial L}{\partial \dot{\theta}} - \frac{\partial L}{\partial \theta} = 0
\end{equation}

\textit{Fill in derivation later\dots} \\

Solving for $\ddot{\theta}$,
\begin{equation}
  \ddot{\theta} = \frac{\dot{\varphi}^2 \sin\theta \cos\theta(I_1 - I_3) - I_3 \dot{\varphi} \dot{\psi} \sin\theta + mgl \sin\theta}{I_1}
\end{equation}

\textit{Fill in derivation later\dots} \\

Solving for $\ddot{\varphi}$,
\begin{equation}
  \ddot{\varphi} = \frac{2(I_3 - I_1) \dot{\varphi} \dot\theta \sin\theta \cos\theta - I_3 \dot{\varphi} \dot{\theta} \sin\theta \cos\theta + I_3 \dot{\psi} \dot\theta \sin\theta}{I_1 \sin^2\theta}
\end{equation}

\textit{Fill in derivation later\dots} \\

Solving for $\ddot{\psi}$,
\begin{equation}
  \ddot{\psi} = \frac{\cos\theta \left[ \frac{(I_1 \sin^2\theta + I_3 \cos^2\theta) \dot{\varphi} \dot{\theta} \sin\theta}{\cos\theta} - 2(I_3 - I_1) \dot{\varphi} \dot{\theta} \sin\theta \cos\theta - I_3 \dot{\psi} \dot{\theta} \sin\theta \right]}{I_1 \sin^2\theta}
\end{equation}

\end{document}
